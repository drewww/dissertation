\chapter{Virtual Spaces}

There was a period in the mid 2000's in which \"virtual worlds\" rapidly rose in popularity. The market research firm Gartner famously predicted in 2007 that by the end of 2011, \"80 percent of active Internet users (and Fortune 500 enterprises) will have a \'second life\', but not necessarily in Second Life.\" \citep{Anonymous:2007wz}. This has not come to pass. Linden Lab has stopped publishing public usage data, but it is clear by comparing its cultural impact to systems like \emph{Facebook}, \emph{Twitter}, and \emph{YouTube} that \emph{Second Life} has been almost completely eclipsed in the public imagination by other types of mediated social experiences. Despite this shift, it is still productive to attend to this period where virtual worlds were viewed as strong potential venues for mediated social interaction. What attracted us to these experiences in the first place? Why did these experiences ultimately fail to catch people's imaginations? Are there aspects of \"virtual\" experiences that were left behind in the shift away from a \"virtual\" approach that might be valuable components of more modern experiences?

In this chapter I will address these questions through an analysis of the properties of virtuality and a pair of design projects: \emph{Information Spaces} and \emph{Presentation Spaces}. Although these projects do not exemplify the best sorts of \emph{in situ} research, they serve as conceptual arguments about how we might think about designing virtual experiences that take advantage of what virtuality offers designers that is missing in other approaches to mediated interaction. This is in contrast to much of the design work in virtual worlds in this period that focused instead on replicating the forms of offline experiences and expecting them to have the same properties when transplanted into a virtual context.

Although in its specifics this argument is perhaps a little dated and unlikely to apply directly in the future (barring a resurgence of virtual world design), it can be seen also as an argument for how to do a close reading of the properties of a platform and letting those properties guide a design that is closely adapted to its tools. Considering virtual worlds in this detail helps highlight some of the taken-for-granted assumptions about non-virtual design.

In a larger sense, a deep discussion of virtuality will also demonstrate how the core research themes of this thesis--grounding, non-verbal actions, and attention--operate in an unfamiliar context. Although readers might be unfamiliar with operating in a virtual world environment, this unfamiliarity will help address these research themes without preconceptions.

I open this chapter with a discussion of what precisely defines a \"virtual world, focusing on the distinction between spatiality and dimensionality. 


% 




I will defer an analysis of why this might have happened for later in this chapter, but 


As with many technology trends, there is a sort of pendulum that swings between different design strategies. Our earliest mediated social experiences (like email and chat) were primarily text-based media. 



This chapter explores some of my early work in the virtual world space and draw comparisons between 


\section{Information Spaces}


\section{Presentation Spaces}


\section{Past and Future Virtuality}