\chapter{ROAR}


% pull some sort of epigraph from mcluhan? not sure exactly what it would be, but something about TV audiences.

The projects described thus far: \emph{Information Spaces}, \emph{Presentation Spaces}, \emph{backchan.nl}, and \emph{Tin Can} are all interested in supporting audiences who are not merely passive receivers of information, but active participants in an experience. My interest in non-verbal actions as a design technique is the main way I accomplish this. This is an important and effective strategy because verbal participation becomes more challenging as the number of potential participants grows. As discussed in Chapter \ref{ch:intro}, the constraint that only one person can talk at a time and the switching costs with synchronous verbal communication impose stiff costs on the engagement level of a group as that group size increases. I have sought in my designs to alleviate these costs by creating other ways to participate that don't have the constraint of seriality, which in turn frees us from the costs of negotiating turn-taking. I have shown the various ways this can create experiences where people feel more engaged with the process, more connected to the audience, and feel like they have an impact on group process in ways that traditional ``being there'' approaches to mediated interaction have trouble providing.


Still, these approaches have limits. As audiences scale up beyond a few hundred (in the case of the largest \emph{backchan.nl} events we've observed), the design approaches we've proposed start to break down. With \emph{backchan.nl} specifically, the first failure mode is that the ``recent'' posts section becomes overwhelmed and hard to keep track of. This causes many users to simply opt out of voting on new posts because they appear faster than they can practically judge them. Larger audiences also bring increased odds of abusive behavior like spamming and mass-voting for low-quality posts. One common approach to handling increasingly large groups is to segment them into manageable chunks where our traditional techniques work well. This works, but it's a bit of a dodge. This chapter will argue that there are ways to simultaneously create compelling small-scale experiences that provide synchronous text-based interaction (e.g. chat) while also providing a series of mechanisms that help a very large group stay aware of each other's moods and interests and engage in various forms of collective activity that make it feel like you're really part of a very large audience.



% mention dunbar, other things?
Size of a collaborative group has always represented a major design challenge. 


By giving participants lightweight ways to engage in the discourse around some shared experience, more people can become involved and feel like they're part of 