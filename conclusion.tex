\chapter{Conclusion}

A major challenge with a design oriented approach to systems research is identifying the contributions that come from the work as a whole. On a per project basis it's easier to demonstrate the achievements and failings of a particular design strategy. But what can we learn from assembling a set of projects in a design space that extends beyond the outcomes of each individual project?

One way to think about this design oriented approach is to consider this thesis a kind of map of the broader design space. Part of the contribution is simply to identify a certain class of designs as being part of a set with common traits and themes. In my work, each of the projects I have described form the core of the design space that I've called complementary communication systems. 

Each of the projects within this larger design space represents an in-depth exploration of a particular area. Within each area, my work can be a concrete guide about what is likely to work well and what is likely to be problematic. The presence or absence of a shared display for \emph{backchan.nl} is one easy example of this kind of guidance. In contexts without a dedicated display, usage of \emph{backchan.nl} falls dramatically. Chapter \ref{ch:virtual} has a wide range of specific tradeoffs inherent to designing virtual interfaces that could provide a guide for designers in virtual worlds. The analysis of \emph{Tin Can} makes a point to show how tablets as a platform operate in a classroom in comparison to laptops. These sorts of findings serve as local directions on the map; areas near the implemented design that we have experimented with and found wanting in various ways.

To try to provide guidance to fellow researchers and designers in the as-yet-unexplored areas of this design space, I have attempted to trace three major design themes through each of the projects. I hope that these can provide broader guidance to others working both in the design space of complementary communication platforms as well on communication systems more generally. Let us now review these themes and the broad conclusions I have drawn regarding each of them.


% talk about mapping at a high level, and place each of the projects in a larger research arc
% reflect on the intentionality of the path

% make some summary statements about the major research themes that cut across all of them

% It is worthwhile at the close of this thesis to return to our three major research themes and consider the lessons we can learn from this set of projects.
\section{Research Themes}

Using text-based communication channels to create a sense of grounding in \emph{backchan.nl} and \emph{Tin Can} had a mixed effect. In neither case did the system create a true sense of the side channel being an automatic component of common ground for people involved in the event; shared displays are not necessarily an automatic route to group-level common ground. Of course, whether a bit of information is part of common ground or not is not a simple binary judgement. Across all the studies we saw varying extents to which people made assumptions about side stage information being mutually visible and understood by others. One of the ways we judged this was how fluidly speakers moved information from the side stage to the main stage.  Situations with single shared displays like optimal \emph{backchan.nl} setups yielded the smoothest transitions between side stage and main stage, while synchronized but non-shared displays in \emph{Tin Can} supported transition, but it took somewhat more negotiation. When the transition goes unremarked, we can conclude that the speaker is assuming that the presence of something on the side stage is mutually understood. While none of the projects reached that state, both \emph{backchan.nl} and \emph{Tin Can} made significant progress in that direction and longer deployments might build a deep enough sense of mutual awareness to best support group grounding. In contrast, in the virtual context the challenge of attending to the side stage inhibits any strong statements about the visualizations role in supporting group grounding.

Non-verbal actions play a critical role in essentially any design of a mediated communication system. All of the projects in this thesis show different ways of constructing those actions yielding different results. Unsurprisingly, actions that are familiar to people because of their widespread use and offline analogs (like voting, which appears in \emph{backchan.nl} and \emph{Tin Can}) are the most successful. The more complex promotion mechanics in \emph{Tin Can} were also well understood, even though there are fewer comparable actions in other mediated experiences. There are also important differences between the explicit voting (like in \emph{backchan.nl} and promotion in \emph{Tin Can}) and implicit voting (like muting in \emph{ROAR}). One way to think about this distinction is that implicit voting mechanics use a user's behavior in a space to intuit their attitudes about someone else or a piece of content. \sidenote{Implicit voting plays a major component in \emph{Facebook}'s decisions about what content to include in someone's feed. Posts from someone whose posts you have frequently clicked on or commented on are more likely to be shown in your feed. Whether the frequency of people clicking on someone's posts influences the frequency of those posts being shown to \emph{other} people (for example, someone who has never clicked on a post from that person) is an open question, but there is some evidence that for brand pages at least, these mechanics are in use.} In contrast, explicit voting asks a user directly what they think or gives them an opportunity to register their feelings through rating or voting. This can be more obviously manipulated by users because the costs or rating or voting something are usually effectively zero.

The allure of implicit voting mechanics is that they usually carry costs which can increase the quality of the signal; if you mute a user, you can't hear their messages anymore. If you click on a link, you have to wait for the page to load. If you leave a comment on something, you cared enough about the content to want to spend the time to say something about it. These costs make these signals more difficult to impersonate effectively and so are often more accurate signals of the value of someone's contributions. The challenge with implicit votes of this form is that they are not necessarily viewed as signals by users, which can make them difficult for people to reason about. It seems intuitive that behavior that isn't visible to others in online spaces (like reading or filtering) shouldn't have an effect on our experience in the space. This hasn't been true for a while, but reminding users of that can be risky.

In general, familiar and explicit mechanics are more quickly understood by users and more complex mechanics take a longer period to become well understood by the group. In both \emph{backchan.nl} and \emph{Tin Can} users tended to have a few hours to become acquainted with the outcome, visibility, and implications of their actions within the context of the system. Adding unfamiliar actions to a system has a similar legibility tradeoff to any UI decision; it imposes a learning cost on users with a potential payoff in enabling new sorts of interactions. These costs can be steeper in a social situation because it is not enough to know what your actions will mean to the system, you must also develop an appreciation for how your actions will be presented to others and understood by them. This is analogous to the mutual awareness challenge in grounding; fluency in the actions of a social system rests on mutual appreciation of the mechanical implications. For systems that are intended be used by transient, temporary groups familiar actions are best. For systems with longer-term aspirations, more nuanced action design becomes possible.

One of the main traditional critiques of backchannel systems had been that they split the attention of the audience. In this argument, the backchannel is viewed as an audience-only distraction that is disrespectful to the presenter. All of the projects in this thesis have pushed back on this notion by demonstrating how shifting to a main stage and side stage model recognizes the possibility of fluid transfer of ideas between a text-based side stage and a ``being there'' main stage. By creating side stages that are designed to influence the main stage in clear, well-defined ways, we can reconfigure the attention problem from being one where attention to anything other than the current speaker is conceived of as inattention, to a situation where attention to the side stage is a different sort of attention that has benefits for the shared experience, broadly understood. In \emph{backchan.nl}, this is achieved by setting up an expectation that questions and comments should be shared through the system and would transition to the main stage at specific moments. In \emph{Tin Can}, the professor's presence in the side stage created a clear sense of the system as part of the classroom space where students could try out speaking turns in a lower stakes environment but one that nonetheless would have an impact on the main stage if others liked their contribution. Thus by building stronger connections between the main and side stages, mediated systems like those I describe are not competing with traditional ``being there'' experiences, but are expanding the ways people can contribute, creating a more inclusive experience, not one where the audience is fragmented.



% grounding - difficult to support, but a useful goal
% actions - critical to non-skeuemorphic social design
% attention - demonstrating situations where mediated tools aren't subtractive, they're addititive

% moving forward!

\section{What's Next?}
Looking forward, I have every reason to believe that this approach to designing communication platforms that complement other collaborative experiences is a growing space. Collaborative applications, broadly construed, have become part of the daily lives of millions of people both in their work and personal lives. We are in the midst of a technical shift as well, that is pushing our traditional hypertext interfaces in the direction of fully-fledged, presence-oriented social spaces. These sorts of systems were challenging to build on traditional web platforms, but the now-broad availability of real-time, event-driven technical infrastructure has made large scale implementations of this vision tractable.

While these spaces don't look like the visions of cyberspace or the metaverse that fiction might have led us to expect, they nevertheless represent a radical change in the frequency and type of engagements with mediated social spaces compared to past tools. My work is situated at the cusp of this transition and seeks to guide future work that attempts to ask the question: ``how can we better support mediated audiences and groups?'' My answer is that by adding thoughtfully designed communication channels to face-to-face or video/audio systems we can enfranchise both large audiences and small groups to take a more active role in the group process. These additional side stages for interaction need not take away from the main stage. Instead, they create alternative roles and modes of interaction that can increase overall engagement and activity in the space.

There are a wide range of domains where these changes can have a big impact. Chapter \ref{ch:roar} shows one kind of approach in an entertainment domain. Online and distance education is another space that is ripe for the addition of these techniques. Work in that space has focused primarily on asynchronous content delivery and self-paced learning. Yet much of the educational value of being in a university setting comes from working in a shared cohort as from attending lectures. \emph{Tin Can} points to how these approaches might operate in a small discussion section, a type of educational space often neglected in online education systems. In a more traditional lecture context, we might look to  \emph{backchan.nl} or \emph{ROAR} as starting points for imagining what a socially engaging, interactive remote lecture might be like. I hope that as these platforms become practical, my work can help guide and support the decisions of their designers and enable more effective participant engagement among groups and audiences of many sizes and different contexts.



% Not yet written! Will just summarize the major contributions in terms of stages, the three research themes (grounding, nonverbal actions, and attention), and point to future directions.