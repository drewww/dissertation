\chapter{Introduction}

% need to open with a 


As researchers first started to build technology to help us communicate with other people at a distance, a consensus on what the goal should be quickly arose: computer mediated communication systems should focus on recreating the experience of being face-to-face with another person. The best system, in this model, is one that seems to disappear in the same way that the best window makes us feel like there is nothing between us and what's on the other side of the glass. Since the early 1970s, it has seemed like we have always been on the verge of a utopian environment where distance disappears and we interact as richly with friends, family, and colleagues around the world as we do with someone sitting in the same room. \citep{Egido:1988vq} And yet, like the paperless office \citep{Sellen:2001uk}, this future has failed to materialize. We can interpret this in two ways: either our tools have failed to deliver on the promise of a ``being there'' level experience or we have been aiming for the wrong target. As with \citet{Hollan:1992tz}, I will argue that the latter case is true. The persistence of a preference for face-to-face communication represents a failure on the part of designers and researchers to recognize both the subtle qualities of face-to-face communication and the potential of computer-mediated-communication tools to complement existing interaction contexts. I will show through a series of design projects and studies where we might find potential ways to create an experience that meets the challenge of being ``beyond being there.''

To introduce this design space, I start by trying to understand the broader context of computer-mediated-communication systems and why work has long focused on the ``being there'' approach to design. I will discuss approaches described in the literature for thinking about the different design strategies one might employ when trying to build new systems with a ``beyond being there'' approach in mind.

% My research has focused on designing, building, deploying, and observing the use of computer mediated communication systems that complement existing communication technologies and face-to-face interaction. In this dissertation, I will introduce a new theoretical framework for thinking about this category of design and trace that theoretical perspective through a series of design projects in different problem domains. 

% might also be able to deploy a ref to Nilles, J., 1988. Traffic reduction by telecommuting: A status review and selected bibliography. Transportation Research Part A: General. Available at: http://linkinghub.elsevier.com/retrieve/pii/0191260788900088.

\section{Being There}
Core to the argument that computer-mediated-communication should simply recreate face-to-face interactions as transparently as possible is the notion that face-to-face interaction is the best we can hope for. Depending on your perspective, it may seem either heretical or obvious to claim that that we might prefer non-face-to-face interaction in certain situations. We can look to research on media selection preferences to convince ourselves that this is true.

Since computer-mediated-communication became technically viable, there has been broad interest in understanding the relative properties of speech, video, and text, as well more esoteric (and largely ignored) modalities like real-time handwriting transmission. \citep{Williams:1977p682} This stream of work can be traced back to \citet{Ochsman:1974vu}, who studied pairs of students coordinating on concrete tasks like scheduling, way-finding, and physical part identification using different sorts of communication tools. This early work focused on measuring which channels were most effective for tasks, and primarily recommended that adding voice was the biggest improvement in performance. This type of work has continued, with researchers expanding their view beyond task performance to trust formation \citep{Bos:2002p256}\citep{Toma:2010p347} and deception \citep{Hancock:2004p314}.

% also cite the chapanis sci amer article? it's better and a little more broad

Much of this work takes place in an experimental psychological tradition, which focuses on controlled lab contexts for studying communication behaviors. As a research context, this leaves much to be desired in studying the complex \emph{in situ} decisions that people make about communication. Outside of lab contexts, we are not assigned a specific tool for a specific task. Instead, we make nuanced and highly contextual choices about what sorts of tools to use in different communication situations. If we accepted the idea that ``being there'' was a primary motivation, we would expect to see people prioritizing tools that were the most like ``being there'' among the available options. This spectrum is also sometimes characterized as ``media richness'' per \citet{Daft:1986p1548}, in which the most rich media are those that are most like being face-to-face and that people will prefer richer tools over less rich tools. Instead, researchers have found repeatedly that in non-lab situations, people frequently choose less rich media over more rich options. \citep{Scholl:2006p210} If richness alone does not predict people's real-life communication media selection decisions, it suggests that other features of a communication medium might also be relevant. 

% there are many other takedowns of media richness theory. Could expand those here.
% try to find some more people who find what scholl finds. I know there's lots but I don't have a handy list right now. 

We can turn to \citet{Brennan:1991wk} as a starting point for other ways we might organize and understand the properties of different sorts of communication media. These aspects of a medium have a major impact on how it's used, and help explain the sorts of results seen in studies like \citet{Scholl:2006p210}.

\begin{description}
\item [Reviewability]{Are records of past interactions easily accessible? Who has access to them?}
\item [Revisability]{How are messages constructed? Do you have time to revise a message before it's sent? Can messages be edited or retracted?}
\item [Synchronicity]{Are messages responded to rapidly, or are there longer gaps between messages?}
\item [Sequentiality]{Does the system support multiple simultaneous conversations, or must all contributions fit into a single shared stream?}
\item [Identity]{How are people represented, and what information is made available about each person?}
\item [Mobility]{In what sorts of spaces can this medium be used? What do we expect about the contexts of other people using the system?}
\end{description}


Since in many contexts people choose non ``being there'' interfaces because of some of these factors, we can safely conclude that designing experiences that seek to be better than ``being there'' will likely take advantage of these sorts of design elements.


% also link to facetime?
The continued focus on ``being there'' designs by much of the communication industry, Cisco's Telepresence systems \ref{fig:cisco-telepresence} and Apple's Facetime \ref{fig:facetime} being major examples reveals a blindness to the potential benefits of mediated communication systems. Furthermore, a focus on this approach hides the many challenges of face-to-face interaction. While in one-on-one situations and very small groups it may be difficult to provide a better experience than face-to-face, I see a wide variety of challenges in interacting with groups that contain more than ten people.  A closer inspection of face-to-face interaction reveals a number of potential challenges:


\begin{itemize}
\item Not all people are equally capable of convincing performances in face-to-face interaction. This can be the result of a variety of factors, including (but not limited to) a lack of confidence in contributing in a specific context, a lack of skill with language, or the impact of a power imbalance in the situation. Many of these can be mitigated in mediated contexts\citep{Siegel:1986ve}, although mediated contexts have their own distinct performance challenges.
\item Simultaneous contribution in face-to-face situations are often viewed as impolite and are generally normatively discouraged. Particularly in large groups, this represents a pressure against contribution and requires a certain amount of overhead to negotiate turn-taking gracefully.
\item Participation in face-to-face contexts usually discloses significant information about someone's identity, while in mediated contexts there are a variety of approaches to limiting disclosure of identity information while still being an active participant.
\item Face-to-face interactions are traditionally ephemeral and difficult to record; mediated interactions are usually quite easy to record. 
\end{itemize}


% I want to be able to cite the commercials, but it's a nightmare finding any information about them that might make them citable like where they appeared, when they appeared, who directed them, etc.


\begin{marginfigure}
	\includegraphics{figures/hole_in_space.jpg}
	\caption{Photos of the Hole in Space exhibit sites in Los Angeles and New York City.}
	\label{fig:hole-in-space}
\end{marginfigure}

\begin{marginfigure}
	\includegraphics{figures/cisco-telepresence.png}
	\caption{Still from a Cisco Telepresence advertisement, centered on connecting an Italian piazza with a Chinese square with a seamless window.}
	\label{fig:cisco-telepresence}
\end{marginfigure}

Despite these challenges, there's still an undeniable magic to the pursuit of recreating face-to-face communication at a distance. This magic is most poetically captured in the famous ``Hole in Space'' \citep{HoleinSpace:1980vn} piece, visually and audibly connecting a storefront in Los Angeles and New York in a way that seemed to make distance disappear. This vision is not simply aspirational, either. Tools to communicate with physically distant people either with audio alone or with an added video connection play a role in the daily lives of millions of people. This desire to experience ``being there'' with someone else is powerful and compelling.


\begin{marginfigure}
	\includegraphics{figures/iphone-face-to-face.png}
	\caption{Still from an Apple advertisement demonstrating the Facetime feature to enable mobile video conferencing.}
	\label{fig:facetime}
\end{marginfigure}




\section{Complementary Communication}
In their famous paper, \citet{Hollan:1992tz} outline an alternative approach which they call ``beyond being there''. They argue that seeking to recreate the experience of ``being there'' was in a way an abdication of our responsibility as designers that left an important design space un-explored. In particular, they urge us to think less about ways to minimize the experience of mediation in communication, but to look instead for ways that mediation can add value to interactions. To take this perspective seriously, we need to shift away from a view of face-to-face interaction as being always better than interactions mediated by technology and instead think critically about potential limitations and challenges with face-to-face interaction and potential benefits that mediation can offer. 

Although Hollan and Stornetta focus on creating mediated experiences that rival or surpass face-to-face experiences, I argue that there is another  strategy that deserves our attention. A certain amount of ``being there'' in the form of audio or video is tremendously valuable, as \citet{Ochsman:1974vu} described in their studies. My work is a kind of compromise, blending elements of ``being there'' with other strategies. I focus on the design of systems that \emph{complement} ``being there'' approaches by attempting to provide the benefits of mediation in situations where either the users are actually physically co-located, or where they are using a traditional ``being there''-type technology (such as audio or video conferencing).

This equivalence may seem unlikely; why should we accept that systems used in coordination with audio or video conferencing would be similar to those used face-to-face? I will argue that a system that can effectively complement face-to-face interaction when its users could simply set it aside and rely on the (presumed superior) affordances of unfettered verbal communication likely has something to tell us about both design and face-to-face interaction more generally. If these systems can provide value in face-to-face contexts, I will show that they also provide value (perhaps even more value) when used to complement systems that seek to create experiences \emph{like} being face-to-face. Furthermore, true ``distributed'' situations are becoming less common. Heterogenous configurations where some people are co-located and others are remote and alone are becoming more common. In these contexts, a system that doesn't operate effectively between co-located users is unlikely to be broadly useful, and would suffer from the disenfranchising effects we see for people who ``dial in'' to a local meeting. \cite{find_a_citation_for_this} Thinking broadly about systems that complement both face-to-face and audio/video sharing will more efficiently lead us to systems effective on both contexts than treating them as separate cases.



% think about listifying this section

% Many of these challenges are common sense, even if they are frequently forgotten when people argue for recreating face-to-face experiences. Face-to-face communication requires relatively explicit turn-taking; multiple speakers in a group make them all largely unintelligible. In mediated environments like chat, simultaneous conversation threads can easily co-exist for long periods of time. There are major identity implications to face-to-face communication. It is difficult to conduct any face-to-face communication without revealing significant information about your identity. In mediated contexts, there are techniques ranging from anonymity to pseudonymity to limit identity disclosing information. Participation in non-mediated interactions is ephemeral, while mediated interaction can easily be archived and represented either in context or after the fact. Participation in face-to-face situations can be limited by confidence, but mediated participation tends to be more disinhibited. \citep{Siegel:1986ve}

% is this a second order effect?
% For a variety of reasons, the power dynamics in social situations are more easily subverted in mediated environments. 

This dissertation is organized around a series of ``primary'' contexts for which I design a particular complementary communication system that enhances the overall experience. Metrics and evaluation strategies vary for each of these pieces, but each project shares a deep interest in trying to fill in the gaps of the ``primary'' interaction space by using the particular strengths of less ``rich'' mediated communication channels. The goal of these interventions is to create environments where people have ways to express themselves non-verbally in addition to whatever existing communication channels exist. By adding mediated communication channels to other existing channels, we can focus on the affordances of each channel to let it do what it does best while addressing the short-comings of each.

% : virtual worlds, face-to-face panel discussions, small group seminar discussions, business meetings, remote information sessions, and live-event spectating

% Part of what's attractive about mediated communication systems is that there is a tremendous variety of ways to design and use them once we set aside a desire to recreate face-to-face interaction. Although in this section I've contrasted mediated communication with face-to-face communication in a way that might imply that mediated communication systems are somehow monolithic and self-similar, the survey of related systems in the section to follow will illustrate the tremendous range of potential systems in this space and demonstrate how thoughtful designs can have widely varying impacts on the experience of communication or collaboration. 

% My work takes this general design strategy of adding new communication channels in a few different directions. In this proposal, I will describe my past work looking at meetings in virtual worlds, audience-speaker interaction in presentations, and classroom discussions. I will also lay out my design for a system to support face-to-face meetings with remote participants. These research contexts vary both in the numbers of simultaneous participants, as well as their geographic configuration. Over the course of my work, I have shifted my attention from configurations where all users are remote (\emph{Information Spaces}) to heterogeneous situations where some or all of the participants are co-located (\emph{backchan.nl}, \emph{Tin Can Classroom}, \emph{Tin Can Conference}). 

% TODO Add a paragraph here that preludes some of the design as research ideas (which will be covered in more depth in their own section) as a way of saying that having these themes is part of what makes this research-worthy. 

\section{Design as Research}
% there are a few points here. 
% 1. a broad introduction to design as a valid research strategy
% 2. projects I did, and the coverage of the space
% 3. themes and theory

Proposing a design space and arguing for its value as an approach to common problems is not, traditionally, the realm of academic research. It is a frequently taken-for-granted assumption at the Media Lab that designing novel technical systems is a natural and defensible way to do research, but outside of that context this may seem like an unusual way to conduct research. Given that this assumption is fundamental to this work, it seems productive to address this question from the start. Design as research is clearly being conducted in a variety of contexts using a variety of methods, yet there is very little discussion or agreement about the fundamental aspects of how that work is conducted and what we can learn from it, let alone a positive argument for why it might be the \emph{best} approach for certain kinds of research questions. It is frequently tolerated but not actively advocated for. 



Because my research practice uses design and development as its primary research method, it seems productive to describe explicitly why this is a valuable research approach. I will describe the sorts of contributions one can make working this way and contrast this approach with other approaches dominant in the fields of human computer interaction, computer mediated communication, and computer supported cooperative work.

% TODO This is a stupid title, especially relative to the section title. FIX IT.
\subsection{Engineering as Research}

% can I get away with these claims? if I wanted to be serious about it I would do a some lit review statistics. but I don't really want to get bogged down in that. Lets see if I can get away without it.
It is a gross simplification, but let us separate work in these fields into three general categories: technology-enabled sociology and psychology, studies of systems in-the-wild, and design of new systems. In the first category, researchers seek to answer the kinds of questions typically of interest of sociologists and psychologists, but deploy technology to allow them to be answer questions have not been able to answer in sufficient detail in the past. This work focuses primarily (but not exclusively) on drawing conclusions about the behavior and experiences of either individuals or collections of people. Studies of systems in-the-wild are, in contrast, more focused on understanding the relationship between the technology and its use, often described as the socio-technical system. Finally, there are researchers who design and build novel systems and then study them. I describe this final category of researchers (in which I consider myself a member) researchers-as-designers. 

These last two categories are deceptively similar. After building a novel system, does one simply run a study on that system like a researcher who didn't build the system themselves? This suggests a critical thought experiment: if the researcher-as-designer could simply imagine a system into existence that looked exactly like the system they wished to study, would that compromise the research in any fundamental way? Put another way, does the actual design and implementation process actually add value to the research or is it just an overhead?

There are two ways to address this issue. First, the thought experiment is subtly mis-framed. Technical artifacts can never really be imagined into existence because their creation is a constant negotiation between the properties of the tools used to create it, the environment in which the design happens, and the reactions the designer has to their own work. In practice, the artifact that comes out of a design process is the result of a lengthy iterative process, even in design processes that don't conceive of themselves as iterative. Simply creating part of an artifact and integrating into another part causes a re-evaluation of those parts in a way that causes designs to drift from their original models. The time spent in the design and implementation processes can be seen as critical for producing a viable design. If we desire to study systems that don't already exist, there is simply no way around spending time on the design itself because our ideas about what the design could or should be before entering those process can't really become real, and if they could they would be unlikely to meet any of the original design goals. From this perspective, we view the development process as a fundamentally necessary cost of creating any novel system. 

% took this from some older writing I did - don't need to dig into that particular account because readers other than Wanda won't be familiar with it.
% The second approach is to turn to Schutlze's confessional account of her dissertation field work in which she struggled with the practical realities of trying to be a participant in the work lives of her informants without injecting her own in-progress conceptualizations of their work experience and without her researcher identity becoming compromised by her affiliation with a specific work group in the organization. Yet, we (I think rightly) value her attempts to not simply sit outside the process she's studying, but to take part in it. This is fundamental to participant-observation as a methodology. 

The second approach is to consider the distinctive values of the design process as a research process. In some fields, we expect that the researcher will become deeply embedded in the process on which their work focuses. In these fields, putting yourself at a distance and insisting that you can simply observe without being part of that process is often viewed as naive. Yet when we shift our focus to creating novel technical artifacts, we prefer to isolate either the users (as in lab studies) or the treat the artifact itself as stable (as in studies of the use of existing artifacts \emph{in situ}). It seems only natural to say that claims about the design of socio-technical systems can be easily augmented by a long-term, rich participation in that exact process. Playing the role of the active participant in the process grants us credibility and real analytical leverage. Not to say that you can't make arguments about design choices of a technology without having been part of the team that made them--you certainly can—-but being a participant in that process provides important insights that we are unlikely to find if we treat the technology as the black box output of an historical design process conducted by others. 


% this paragraph needs to be about WHAT ARE THE INSIGHTS FROM THE DESIGN PROCESS?

It is difficult to precisely identify the sorts of contributions that would not be possible without engaging in the design process because there are few examples of a team doing novel design work and handing off the result to a separate researcher and comparing their results. We do have a large body of research on systems conducted by non-designer researchers, but there's nothing to systematically compare it to. I hope that my work highlights how the researcher-as-designer can operate effectively in both roles and serve as a starting point for a broader discussion about why designing and studying systems is as valuable a research strategy as studying existing systems. \sidenote{Designing and building systems can take a substantial amount of time, especially if you hope to deploy those systems \emph{in situ} instead of in lab contexts. If papers are the main output metric for a researcher, this approach is not necessarily an efficient way to generate papers, since few conferences will accept papers on the design or development of a novel system alone. This might help explain the waning popularity of this sort of research.}

If we accept that conducting design and development are valuable research processes, we must consider the challenges to this kind of work. If we hope to avoid the limits of studying design work in decontextualized lab situations, then we need to find situations were our system might credibly be used ``for real.'' The best situations are ones in which people interacting with and through the system can do so in the normal contexts in which they might interact with such a system: using their own devices, in places that are familiar to them, and with people who they might normally use such a system with. These contexts can be quite difficult to secure. Some systems require a certain scale to reveal meaningful results; had a researcher designed Twitter, they would have been very hard-pressed to find a context in which they could study it in a legitimate way. These constraints are also acute when designing for business contexts. Deploying research software in business contexts poses risks for the business in terms of data security as well as ethical concerns about businesses compelling their employees to use the system. Although this limits the kind of design work the researcher-as-designer might credibly study, these constraints are notably different than the constraints on researchers who study existing in-the-wild systems. In many ways, these biases are nicely complementary:

\begin{table}[tb]
\begin{tabular}{r|l}
\textbf{Researcher-as-Designer} & \textbf{Researcher} \\
using novel technology & using existing technology \\
not widely deployed or available & popular/widely used \\
smaller, bounded user groups & larger, fluid user groups \\
extrinsically motivated users & intrinsically motivated users\\
rougher edges & well polished\\
consumer-oriented & consumer or professional\\
bounded use durations & potentially unbounded use durations\\
internal process traces & publicly observable process traces\\
\end{tabular}
\label{table:research-focus-comparison}
\caption{Comparison between the kinds of technical contexts that the researcher-as-designer typically studies compared to the researcher.}
\end{table}

On most of these axes, it is not that researchers are incapable or uninterested in studying the kinds of systems that the researcher-as-designer studies, but that (for a variety of systemic and historical reasons) they have gravitated towards these themes. Whatever the reason, these differences add to the value of the researcher-as-designer approach. Even if taking the role of researcher-as-designer takes a substantial time investment to create the systems being studied, if it reveals insights about new kinds of systems that traditional researchers might not pay attention to, then it can be a valuable approach. This is most true with respect to the properties of the technology in the system. The researcher-as-designer can also be conceived of as mapping terrain that the designer-as-professional and other researchers will later cover when the technology becomes more widely accessible or viable. Furthermore, the kinds of deep data collection possible with custom-engineered systems open up a variety of analytical options that are often not possible when trying to collect data from publicly available and usually corporate-controlled contexts like Facebook or Twitter.

\section{Methodology}
% ??? Could talk about studies in authentic contexts and do the whole critique of lab-based studies here. Probably not super relevant, and a bit risky since a lot of the stuff I'll end up talking about doesn't have particular studies attached to it. 

% other potential topic: broad ideas about how people and technology interact and what the focus of study is? 

\section{Design Spaces, Themes, and Theory}

One of the challenges of building technical systems as research is understanding the scope of conclusions. If you took a particular design element into a different system, would it operate in the same way? What are the relationships between the sorts of people using the system and the socio-technical structures that emerged? These are difficult questions to answer within the scope of a single project. The researcher may have solid intuition, but the tendency of the researcher is probably to see overly-general results more often than overly-specific results. One of the ways I address this is by describing a series of projects in this design space and examining design elements and themes in a variety of contexts.

This would be less effective as an approach if each of the design spaces was quite similar. Instead, the design spaces covers a range of design contexts:

\begin{description}
\item[Main Stage]{The medium for the main stage, e.g. the site of the primary shared experience of the audience.}
\item[Shared Display]{The presence or absence of a shared display.}
\item[Side Stage Attention]{The frequency of audience attention on the side stage. This is quite qualitative and varies across users, but each system embeds contains certain assumptions about the relative importance of the side stage to the main stage.}
\item[Audience Size]{The target audience size for the system.}
\end{description}

\begin{table*}[tb]
	% \centering

\begin{tabular}{r|llll}
& Main Stage & Shared Display & Side Stage Attention & Audience Size \\
\textbf{backchan.nl} & face-to-face & yes & infrequent & 20-500 \\
Information Spaces & virtual world & yes & infrequent & 10-20 \\
ROAR & broadcast video & no & frequent & > 1,000 \\
\textbf{Tin Can} & face-to-face & no & occasional & 10-20 \\
\end{tabular}
\label{tab:project-axes}
\caption[][15pt]{Comparing the projects covered in this dissertation on the major axes that distinguish them. Projects in bold are major projects discussed in the most depth. Small project variants (e.g. backchan.nl for remote Q\&A, Tin Can Meetings, etc.) are not included.}
\end{table*}

% could I identify a set of common design elements that can be traced across all projects?
% voting
% chat
% shared display?

Table \ref{tab:project-axes} lists the research contexts covered in this dissertation, and describes their location on the major context axes. The variety across these axes helps show the breadth and scope of my work. There are, of course, significant limits; the design space is focused only on designing systems that complement an existing communication experience and the design elements used in these systems are relatively focused.


Across this set of projects, I trace three major research themes:

\begin{description}
	\item[Grounding]{My work uses shared displays in a variety of different capacities. I contend that these kinds of public displays can play a powerful role in helping to ground, in the \citet{Clark:1989uc} sense, a conversation. In particular, shared displays can provide ways to non-verbally acknowledge discourse presentations. By their very shared nature, the contents of shared displays might accelerate the creation of common ground. The different ways that these shared displays operate in my work helps provide insight into both particular design techniques to support grounding as well as the broader discussion around how common ground operates in mediated communication contexts.}
	\item[Non-verbal actions]{As a result of the drive to create a sense of ``being there'', mediated interaction systems failed to consider the ways that we communicate non-verbally, assuming that higher fidelity video and audio would be sufficient to capture that communication. I contend that our non-verbal actions in the physical world are a critical component of body language, and when creating mediated channels we should strive to create new vocabularies of action that enable people to communicate non-verbally. In much the same way that in a shared physical space we can observe people interact with objects around us, so too should people's actions in mediated systems be visible and part of supporting a sense of presence and awareness. How these action vocabularies are constructed and communicated to people is critical to the success of these sorts of systems.}
	\item[Attention]{Adding communication channels forces us to decide how we manage our attention. Which channels do we attend to? How is our attention made visible to others, and how does it affect their impressions of us? Furthermore, how we decide which channels are appropriate for which kind of communication? Although attention is not an inherent design issue, understanding how people think about and enact attention in situations with multiple available communication channels is critical to designing appropriate options and understanding the practices that evolve around them.}
\end{description}

This work contributes on two levels. First, by creating and deploying interfaces with particular properties, I provide concrete guidance and insight about particular specific design strategies and interfaces. This is valuable for designers and researchers thinking about how they design this variety of communication systems. I also contribute to the broader discourse about the three research themes laid out above. In each case there are both broader theoretical contributions to be made as well as specific findings that contribute to scholarly discussions about these issues.



\section{Overview}
Introduce the following chapters. Write this at the end.