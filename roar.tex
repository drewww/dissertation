\chapter{ROAR}


% pull some sort of epigraph from mcluhan? not sure exactly what it would be, but something about TV audiences.

The projects described thus far: \emph{Information Spaces}, \emph{Presentation Spaces}, \emph{backchan.nl}, and \emph{Tin Can} are all interested in supporting audiences who are not merely passive receivers of information, but active participants in an experience. My interest in non-verbal actions as a design technique is the main way I accomplish this. This is an important and effective strategy because verbal participation becomes more challenging as the number of potential participants grows. As discussed in Chapter \ref{ch:intro}, the constraint that only one person can talk at a time and the switching costs with synchronous verbal communication impose stiff costs on the engagement level of a group as that group size increases. I have sought in my designs to alleviate these costs by creating other ways to participate that don't have the constraint of seriality, which in turn frees us from the costs of negotiating turn-taking. I have shown the various ways this can create experiences where people feel more engaged with the process, more connected to the audience, and feel like they have an impact on group process in ways that traditional ``being there'' approaches to mediated interaction have trouble providing.


Still, these approaches have limits. As audiences scale up beyond a few hundred (in the case of the largest \emph{backchan.nl} events we've observed), the design approaches we've proposed start to break down.\sidenote{This is not specific to mediated interactions; interactions between unmediated communities change dramatically as the size of the community goes up, too. \citep{dunbar_number}} With \emph{backchan.nl} specifically, the first failure mode is that the ``recent'' posts section becomes overwhelmed and hard to keep track of. This causes many users to simply opt out of voting on new posts because they appear faster than they can practically judge them. Larger audiences also bring increased odds of abusive behavior like spamming and mass-voting for low-quality posts. One common approach to handling increasingly large groups is to segment them into manageable chunks where our traditional techniques work well. \sidenote{Often this segmentation happens for technical reasons as much as social reasons. Maintaining a sense of presence in a mediated group tends to be an $N_2$ scaling problem. The scaling factor is particularly brutal for virtual worlds which tend to have problems even rendering large numbers of avatars, let alone managing the communication problems of letting them interact. This has let many systems simply avoid the problems of large scale interaction because they were technically unrealistic.} This works, but it's a bit of a dodge. This chapter will argue that there are ways to simultaneously create compelling small-scale experiences that provide synchronous text-based interaction (e.g. chat) while also providing a series of mechanisms that help a very large group stay aware of each other's moods and interests and engage in various forms of collective activity that make it feel like you're really part of a very large audience. Thinking about mediated crowds in this way brings up compelling conceptual questions like: 

\begin{enumerate}
\item{How do people find groups of people to talk with?}
\item{Do collective activities like chanting or doing the wave have online crowd analogs?}
\item{How do you manage antisocial behavior in online crowds?}
\item{How can you create opportunities for deeper engagement with the event that have an impact on other audience members' experiences?}
\end{enumerate}

Constructing a sense of remote viewership is not a new activity. As radio, film, and television broke down removed the physical constraints of audiences and performers being co-located, we were able to create enormous audiences all experiencing something together. Yet there was clearly something important about the presence of the audience. We still go to movie theaters to watch movies together, and TV shows frequently have live audiences (or simulate live audiences with a laugh track) to try to foster a sense of experiencing something with someone else.

Over time, the structure of these events has even evolved to take advantage of technology to create a sense of engagement and involvement. Shows like \emph{American Idol} use text messages to allow audiences to vote for specific contestants. This is a relatively thin form of engagement: feedback is quite delayed, votes are essentially anonymous, and the pool of votes is huge which makes it hard to feel like you're making a difference. Nevertheless, this is part of a long-term campaign on the part of broadcasters to try to make it fee like broadcast television isn't simply receiving data, but trying to bring back the historical experience of being in a crowd with other viewers.

In this chapter, I will describe a system called \emph{ROAR} that tries to develop the design techniques described and studied in past chapters towards audiences of extremely large scale. I will talk about related work in the social TV space as well as discuss the other sorts of tools that people use to create similar sorts of experiences. Finally I will describe the major components of ROAR's design: sections, pulse, shouts, and feedback. 

Unlike the main projects in this thesis, \emph{backchan.nl} and \emph{Tin Can}, I have not done any deployments of \emph{ROAR}. This is primarily for practical reasons. It is quite difficult to reach audiences of the sizes for which \emph{ROAR}'s design is specialized. Organizations with audiences of this size tend not to be interested in trying prototype code during a large scale event. Still, these sorts of explorations can provide valuable guidance about a design space that we might otherwise overlook because it is prohibitively difficult to deploy prototype code. As a capstone to a series of more elaborately studied systems, I view this chapter as a forward-looking description of future work that can show how the principles developed in earlier chapters could grow to address the needs and interests of larger public audiences. The other major shift in this chapter is a desire to focus on experiences that are feasible now, not speculative experiences 



% on co-located audiences, broadcasters discovered the challenges of 
% 
% made it technically possible to assemble remote audiences, we were left 
% 
% 
% The transition from co-located audience to remote audience was an uncomfortable one. As audiences 
% 
% Broadcast media like radio and television created the modern audience. More than any of the other technologies described in this thesis, only broadcast media have created truly enormous simultaneous audiences. Broadcasting live events broke down the physical constraints involved in creating a space that could support large local audiences. It became possible to share at least a portion of the experience of watching something live while being at a distance. 

% talk about how spaces evolved to support the remote audience. cameras as the eyes of the masses. remove viewers giving experience meaning. shift to small ``studio'' audiences. the laughtrack and roar of the crowd to give meaning to the remote experience. simulating an audience.

% trace history of how events evolved to become better aware of the remote audience? incresed value of space in the remote visual field (e.g. advertising that shows on screen only, not in the physical space - provides local addressing). 



% topics in a random order (for now)
% scaling non-verbal actions (distribution approaches, viral spread mechanics, etc)
% creating small scale interactive groups
% 	can put the twitter arguments here - why having bounded audiences is good + how to discover those audiences
% synthesizing interaction
% 	talk about the chat visualization systems.
% 
% how to organize this chapter?
% 	one model is to march through a series of designs and talk about the evolution. In this model we would do:
%	original trending words view
%	first prototype
%	second prototype visualizations
%	final version
%
% the challenge is a lot of the pieces are more conceptual and don't necessarily "exist" in any of these prototypes. shouts and questions and votes and betting are all good examples of that. this is an argument for laying out the basic strategies in a conceptual way and then moving through the concrete prototypes with those in mind. Treat the prototypes as instantiations of those visions and just say "oh, this is what changed". 

% so final strategy is:
%	introduce the field
%	talk about the history of audiences (maybe, this isn't my strong point)
%	talk related work (put the why not twitter, why not facebook section here)
%	set out the values here
%	series of modular approaches: sections, pulse, shout, voting/betting, 
%   future directions (representations in the physical arena, creating new streams of content, tumblr-like engagement, etc)


\section{Related Work}

% really need to do at least a cursory walk through the CHI/CSCW literature on social tv stuff. I know frank did stuff like this back in the day.
%
% A Visual Backchannel for Large-Scale Events

\subsection{Alternatives}

There is a widely-held assumption that interaction around live events like those that \emph{ROAR} aims to support are already taking place on existing social networks like \emph{Facebook} and \emph{Twitter}. If this were true, it would suggest a very different approach; one that looks much more like \emph{Visual Backchannel}. However, I will argue in this section that the social spaces created by \emph{Facebook} and \emph{Twitter} are not (and barring pretty fundamental changes to their mechanics can not become) spaces for truly interactive discussions about a live event.

As I've argued about grounding elsewhere in this thesis, creating social situations with mutual awareness is critical to creating effective conversation spaces. If there is any concern that others can't see what you're saying, it breaks the cycle of awareness that is critical for moving a conversation forward. The publish/subscribe model adopted by and \emph{Twitter} gets in the way of this once conversations scale beyond just one person. If Bob follows Alice, and Charlie follows Alice but not Bob, Charlie will see only parts of a conversation that are not directed at Bob with an ``@'' reply. \sidenote{I'm going to mostly ignore \emph{Facebook} for the sake of simplifying the argument. The outlines of the argument are similar, but \emph{Facebook}'s more integrated reply format avoids some of these problems. But the lack of reliability of posts being seen by followers (as few as 20\% of your followers see any given post \citep{facebook_post_seen_freq}) adds another challenge to creating a grounded community.}

The nominal solution for this is hashtags, e.g. words prefixed with a ``\#'' character. These mark a tweet as part of a larger set of tweets on a theme, and places that tweet in a stream in the so-called ``discover'' part of the interface. This interface has a similar grounding-inhibiting design. First, it is impossible to tell if others are actively viewing the hashtag in the ``discover'' mode. Whether or not that audience exists is essentially unknowable to posters. Hashtags are visible amongst follower networks (i.e. you can see hashtags that seem to be popular within your network and be involved in this conversations) but there is little evidence that participations cross existing follower/following relationships.

The problem of an audience for tweets with a hashtag is compounded because tweets with a hashtag do have one guaranteed audience: your existing followers. Every tweet, hashtagged or not, is inserted in the streams of all your followers. When balancing a potential imagined audience and a guaranteed one, people will write for the audience they know they have. For major events like the Superbowl or World Cup, we can safely assume that our audience is likely to contain many people who are watching the same event. But even for moderately less popular events, it is far more likely that most of our followers are not co-watching and are unlikely to be deeply interested in a detailed conversation about what's happening. 

We can see that this is true by looking at the sorts of tweets that people write to hashtags. We take as our 

% do an informal analysis of tweets for the latest ML talks event. categorize into announcements, quotes, other https://twitter.com/#!/search/realtime/mltalks

This informal analysis is complemented by summary data from Bluefin Labs, a company that captures and analyzes \emph{Twitter} conversations about TV shows. For each show they analyze, they report the total number of tweets about the show and the total number of people who tweeted about the show. The ratio of these two values is the average number of tweets per tweeter. For the vast majority of shows they track, this value is between $1$ and $2$; the average viewer tweets at most twice about a single show. This holds true even for sports matches that can last 2-3 hours. It is intuitive that only one or two messages per person is a strong indication that there is not significant conversation going on; conversations simply cannot take place in one or two messages. Nevertheless, it is useful to compare these participation patterns with a social context that looks more like \emph{ROAR} to argue for \emph{ROAR}'s design approach.

% put in a screenshot of the blue fin data page

Although we don't have data for people using a system like \emph{ROAR} yet, we can look to comparable experiences that already exist. Many sports fans use platforms like Internet Relay Chat to talk about live events. Based on logs collected over the course of two weeks, we can calculate a metric similar to that reported by Bluefin Labs for conversations on \emph{Twitter}. We can't easily link specific messages to specific events, but we can measure how many messages each person sends per hour in which they send any messages at all; essentially for people who are chatting in a given window, how many messages do they send? This approximates the tweets per unique author per show metric. 

This measure varies pretty substantially between which chat room you collect data from. Data is reported in Table \ref{tab:chat_message_rates}. 


% channel name
% average unique hourly users (average )
% total messages  \sigma{msg}
% total duration t_n - t_0
% average messages per active user
% comments

\begin{table*}[tb]
\begin{tabular}{r|cccccl}
\textbf{Channel} & \textbf{Total Messages} & \textbf{Total Users} & \textbf{Mean Hourly Active Users} & \textbf{Mean  Messages Per Active User-Hour} \\

\#reddit-soccer & 20317 & 251 & 7.9 (SD=9.9) & 16.6 (SD=25.5) \\
\#football & 11936 & 199 & 8.8 (SD=8.2) & 10.1 (SD=11.8) \\
\#teamliquid & 110910 & 1940 & 29.7 (SD=19.2) & 9.5 (SD=18.1) \\
\#joindota & 93010 & 9022 & 52.8 (SD=113.3) & 3.4 (SD=9.6) \\
	
\end{tabular}
\label{tab:chat_message_rates}
\caption{Comparison of participation rates across different IRC chat rooms.}
\end{table*}

The most direct comparison to \emph{Twitter} is the final numeric column on the right. Mean messages per active user-hour represents the average number of chat message a user sends in hours where they send any messages at all. If every user logged on, sent one message, and then stayed silent, this metric would be 1. It would also be 1 if every user sent one message per hour. As a result, the floor of this metric (as with the tweets / unique author metric) is 1. Larger values represent users who tends to send many more messages during hours where they are chatting at all. This metric is artificially depressed relative to the \emph{Twitter} metric because our data covers not just moments when there are live events going on, but 24 hours a day. This is captured by the variance in the fifth column: the number of active chatters varies widely between active moments and inactive moments. Inactive moments don't exist in the \emph{Twitter} dataset because it focuses on only the tweets associated with a television show. Despite this, we see values ranging from 3.4 to 16.6, compared to \emph{Twitter}'s 1-2. This suggests that chat contexts elicit a 200-1600\% increase in per-active-user participation. 

The large variance in participation rates seems (in this very preliminary review) to be correlated with the size of the rooms. The rooms with the highest per-user participation were also the rooms with the fewest active users. This supports the \emph{ROAR}'s core argument that smaller interaction contexts will be more interactive than larger undifferentiated contexts. This might also explain part of the difference between \emph{Twitter} and a chat-style system: \emph{Twitter} looks much more like a single large undifferentiated room than a focused small scale social space.




% We can see th

%The easiest way to understand that is does is exist would be getting a reply from someone who doesn't follow you who discovered your tweet via the ``discover'' tab. Although this is possible, a casual perusal of tweets involved in trending topics shows vanishingly few replies to other tweets; RTs are far more common. These impediments to mutual visibility 

% would be interesting to make an empirical argument about how many @ replies are to people via hash tags. could try to do this by looking at @ that were replied to that were between people who don't follow each other and have used a hash tag in a period. that's a bit elaborate, though. 

% make a diagram of the alice/bob/charlie system
% put a screenshot of the discover interface in here in




